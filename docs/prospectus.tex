\documentclass {article}
\usepackage{verbatim}
\usepackage{setspace}
\doublespacing
\begin {document}
\section {CS400 Project}
\subsection{Abstract}

\paragraph{}
Our team proposes to implement a compiler that accepts a language of
our own design targeting the 65816 microprocessor, and thus the 65xx
family by extensions.  The language will have C-like constructs and
Lisp-like syntax.  To demonstrate the viability of our compiler, we
will write an application for the Super Nintendo Entertainment System
(SNES) using the tools we develop.

\subsection{Design Decisions}
\subsubsection{Using C}

\paragraph{}
The C programming language is ideal choice as a source language for
several reasons.  First, it has a fairly direct translation into
assembly language.  In fact, C is often described as a portable
assembly language.  Because of its close to one-to-one correspondence
with assembly, and because it allows bit manipulation and direct
access to memory, code written in C compiles into fast executable
code.  Second, C is a real world language.  While it is often
described as a "systems language", it finds uses in scientific
computing, gaming, and even applications programming. Because of the
limitations of the 65816 architecture, C is the standard choice for an
implementor that wants to write applications in a "real" high level
language in an embedded environment. Finally, C is a small language.
The entire language (C99) only has thirty-seven keywords.  These
reasons make C an ideal language for a semester long compiler
construction project.

\subsection{Why 65816?}

\paragraph{}
It is fair to ask why the 65816 processor was chosen as the target
language for our compiler.  After all, the processor is nearly 25
years old, and has several limitations which make writing code in a
high a level language difficult.  The primary considerations that went
into our decision to use the 65816 architecture are as thus: (1) There
continues to be an active development community that produces SNES
games.  Unfortunately, there are no free C (or other high level
language) compilers available to this community.  Commercial compilers
exist, but they are not actively maintained.  (2) Our teammate, Parker
Trofatter, has implemented an assembler for the 65816 architecture,
developed games for the SNES, and as such, has an expert level
knowledge of the architecture. Furthermore, the limitations of the
architecture force us to make design compromises and innovations that
make compiler hacking fun.

\subsubsection{Using Lisp Syntax and Lisp as an Implementation Language}

\paragraph{}
Using Lisp syntax instead of C syntax has many advantages.  First it
basically eliminates the difficult (and very boring) parsing step.
Writing a Lisp parser is utterly trivial.  Furthermore, since we are
using Common Lisp as our implementation language, the entire parser is
the following code:

\begin{verbatim}
    (read)
\end{verbatim}

\paragraph{}
Not only is the parsing easy, but the data structure that
s-expressions represent work great as an abstract syntax tree.  Common
Lisp is basically made specifically for this kind of work, so it
includes lots of extremely convenient code for manipulating code in
this way.  S-expressions look awesome.

\paragraph{}
Basically, using s-expressions saves us a ton of work, but it does
much more than that.  We can expose the full power of Common Lisp as a
sort of advanced ``preprocessor.''  With a system like this, the user
can implement arbitrarily complex language features by writing Common
Lisp macros (mixed with his code), and then using them.  For example,
it would be pretty easy to add a simple object system.  I've even
hacked out some code for this, but it's currently too primitive to
show.

\paragraph{}
We can also use this ``preprocessor'' (it's really much more than
that) to implement a lot of the compiler itself.  Almost all Common
Lisp implementations actually implement very large subsets of the
language in straight-up lisp code using macros.  For example, the
looping constructs, the entire object system, etc, are almost always
implemented in lisp with macros.

\subsection{Hardware Issues}
\paragraph{}
There are several quirks of the 65816 architecture which provide
challenges for compiler construction.  First, there is only one
general purpose register the 'A' or 'accumulator' register.  Second,
memory is divided into several 'banks', and while data can cross bank
boundaries, code cannot.  There are two jump subroutine operations:
JSR and JSL.  JSR is used when jumping to code inside of the current
bank, while JSL is used to jump to code in other banks.  As such, code
must be organized in such a way that as few JSL's as possible are
used.  There are several different types of RAM in the 65816
architecture, and they are accessed differently, and used for
different purposes.  It is difficult to define general purpose memory
access mechanisms.  As mentioned, RAM is located inside several banks.

\subsection{Simple Compilation Examples}
\subsection{Implementation Strategy}

\paragraph{}
Right now we have a working compiler for a very small subset of the C
language.  We support subroutines, label's, goto's, variable scopes,
and ifs; We are very close to having argument passing, basic
arithmetic, and more.  The compiler is currently simple, single-pass,
and non-optimizing, so we can get away with implementing the entire
system as lisp macros that expand into code that, when invoked, prints
out assembler code.  Eventually, for optimization and more complicated
features, we will probably need to store the code in a data structure
and perform different sets of transformations in passes.

\end{document}
